% Options for packages loaded elsewhere
\PassOptionsToPackage{unicode}{hyperref}
\PassOptionsToPackage{hyphens}{url}
%
\documentclass[
]{article}
\usepackage{amsmath,amssymb}
\usepackage{iftex}
\ifPDFTeX
  \usepackage[T1]{fontenc}
  \usepackage[utf8]{inputenc}
  \usepackage{textcomp} % provide euro and other symbols
\else % if luatex or xetex
  \usepackage{unicode-math} % this also loads fontspec
  \defaultfontfeatures{Scale=MatchLowercase}
  \defaultfontfeatures[\rmfamily]{Ligatures=TeX,Scale=1}
\fi
\usepackage{lmodern}
\ifPDFTeX\else
  % xetex/luatex font selection
\fi
% Use upquote if available, for straight quotes in verbatim environments
\IfFileExists{upquote.sty}{\usepackage{upquote}}{}
\IfFileExists{microtype.sty}{% use microtype if available
  \usepackage[]{microtype}
  \UseMicrotypeSet[protrusion]{basicmath} % disable protrusion for tt fonts
}{}
\makeatletter
\@ifundefined{KOMAClassName}{% if non-KOMA class
  \IfFileExists{parskip.sty}{%
    \usepackage{parskip}
  }{% else
    \setlength{\parindent}{0pt}
    \setlength{\parskip}{6pt plus 2pt minus 1pt}}
}{% if KOMA class
  \KOMAoptions{parskip=half}}
\makeatother
\usepackage{xcolor}
\usepackage[margin=1in]{geometry}
\usepackage{color}
\usepackage{fancyvrb}
\newcommand{\VerbBar}{|}
\newcommand{\VERB}{\Verb[commandchars=\\\{\}]}
\DefineVerbatimEnvironment{Highlighting}{Verbatim}{commandchars=\\\{\}}
% Add ',fontsize=\small' for more characters per line
\usepackage{framed}
\definecolor{shadecolor}{RGB}{248,248,248}
\newenvironment{Shaded}{\begin{snugshade}}{\end{snugshade}}
\newcommand{\AlertTok}[1]{\textcolor[rgb]{0.94,0.16,0.16}{#1}}
\newcommand{\AnnotationTok}[1]{\textcolor[rgb]{0.56,0.35,0.01}{\textbf{\textit{#1}}}}
\newcommand{\AttributeTok}[1]{\textcolor[rgb]{0.13,0.29,0.53}{#1}}
\newcommand{\BaseNTok}[1]{\textcolor[rgb]{0.00,0.00,0.81}{#1}}
\newcommand{\BuiltInTok}[1]{#1}
\newcommand{\CharTok}[1]{\textcolor[rgb]{0.31,0.60,0.02}{#1}}
\newcommand{\CommentTok}[1]{\textcolor[rgb]{0.56,0.35,0.01}{\textit{#1}}}
\newcommand{\CommentVarTok}[1]{\textcolor[rgb]{0.56,0.35,0.01}{\textbf{\textit{#1}}}}
\newcommand{\ConstantTok}[1]{\textcolor[rgb]{0.56,0.35,0.01}{#1}}
\newcommand{\ControlFlowTok}[1]{\textcolor[rgb]{0.13,0.29,0.53}{\textbf{#1}}}
\newcommand{\DataTypeTok}[1]{\textcolor[rgb]{0.13,0.29,0.53}{#1}}
\newcommand{\DecValTok}[1]{\textcolor[rgb]{0.00,0.00,0.81}{#1}}
\newcommand{\DocumentationTok}[1]{\textcolor[rgb]{0.56,0.35,0.01}{\textbf{\textit{#1}}}}
\newcommand{\ErrorTok}[1]{\textcolor[rgb]{0.64,0.00,0.00}{\textbf{#1}}}
\newcommand{\ExtensionTok}[1]{#1}
\newcommand{\FloatTok}[1]{\textcolor[rgb]{0.00,0.00,0.81}{#1}}
\newcommand{\FunctionTok}[1]{\textcolor[rgb]{0.13,0.29,0.53}{\textbf{#1}}}
\newcommand{\ImportTok}[1]{#1}
\newcommand{\InformationTok}[1]{\textcolor[rgb]{0.56,0.35,0.01}{\textbf{\textit{#1}}}}
\newcommand{\KeywordTok}[1]{\textcolor[rgb]{0.13,0.29,0.53}{\textbf{#1}}}
\newcommand{\NormalTok}[1]{#1}
\newcommand{\OperatorTok}[1]{\textcolor[rgb]{0.81,0.36,0.00}{\textbf{#1}}}
\newcommand{\OtherTok}[1]{\textcolor[rgb]{0.56,0.35,0.01}{#1}}
\newcommand{\PreprocessorTok}[1]{\textcolor[rgb]{0.56,0.35,0.01}{\textit{#1}}}
\newcommand{\RegionMarkerTok}[1]{#1}
\newcommand{\SpecialCharTok}[1]{\textcolor[rgb]{0.81,0.36,0.00}{\textbf{#1}}}
\newcommand{\SpecialStringTok}[1]{\textcolor[rgb]{0.31,0.60,0.02}{#1}}
\newcommand{\StringTok}[1]{\textcolor[rgb]{0.31,0.60,0.02}{#1}}
\newcommand{\VariableTok}[1]{\textcolor[rgb]{0.00,0.00,0.00}{#1}}
\newcommand{\VerbatimStringTok}[1]{\textcolor[rgb]{0.31,0.60,0.02}{#1}}
\newcommand{\WarningTok}[1]{\textcolor[rgb]{0.56,0.35,0.01}{\textbf{\textit{#1}}}}
\usepackage{graphicx}
\makeatletter
\def\maxwidth{\ifdim\Gin@nat@width>\linewidth\linewidth\else\Gin@nat@width\fi}
\def\maxheight{\ifdim\Gin@nat@height>\textheight\textheight\else\Gin@nat@height\fi}
\makeatother
% Scale images if necessary, so that they will not overflow the page
% margins by default, and it is still possible to overwrite the defaults
% using explicit options in \includegraphics[width, height, ...]{}
\setkeys{Gin}{width=\maxwidth,height=\maxheight,keepaspectratio}
% Set default figure placement to htbp
\makeatletter
\def\fps@figure{htbp}
\makeatother
\setlength{\emergencystretch}{3em} % prevent overfull lines
\providecommand{\tightlist}{%
  \setlength{\itemsep}{0pt}\setlength{\parskip}{0pt}}
\setcounter{secnumdepth}{-\maxdimen} % remove section numbering
\ifLuaTeX
  \usepackage{selnolig}  % disable illegal ligatures
\fi
\IfFileExists{bookmark.sty}{\usepackage{bookmark}}{\usepackage{hyperref}}
\IfFileExists{xurl.sty}{\usepackage{xurl}}{} % add URL line breaks if available
\urlstyle{same}
\hypersetup{
  pdftitle={BIS\_015L\_Mid\_review},
  pdfauthor={Zhuoya Wang},
  hidelinks,
  pdfcreator={LaTeX via pandoc}}

\title{BIS\_015L\_Mid\_review}
\author{Zhuoya Wang}
\date{2024-01-30}

\begin{document}
\maketitle

\hypertarget{lab-2.1-objects-classes-nas}{%
\section{Lab 2.1 Objects, Classes \&
NAs}\label{lab-2.1-objects-classes-nas}}

\begin{Shaded}
\begin{Highlighting}[]
\DocumentationTok{\#\#\#\#\# Lab 2}

\CommentTok{\# sum(A, B, C)}
\CommentTok{\# Mean(c(A, B, C)) \# ABC are the object names}
\CommentTok{\# class() \# check the type of object}
\CommentTok{\# is.xx() \# check if the type of object is xx }\AlertTok{\#\#\#}\CommentTok{ output: True or False}
\CommentTok{\# as.xx() \# create an object specified as xx}

\DocumentationTok{\#\#\#\#\# Missing Values}

\CommentTok{\# is.na() \# check every element in an object (number of TF = \# of elements)}

\CommentTok{\# anyNA() \# check if the object has NA inside (only T/F)}

\CommentTok{\# mean(objectname, na.rm = T/F) \# caculate the mean of object, na.rm = remove NA}

\DocumentationTok{\#\#\#\#\# Vectors}

\CommentTok{\# my\_vector \textless{}{-} c(10, 20, 30) \#numeric vector}

\CommentTok{\# days\_of\_the\_week \textless{}{-} c("Monday", "Tuesday", "Wednesday", "Thrusday", "Friday", "Saturday", "Sunday") \# Characters always have quotes}


\CommentTok{\# days\_of\_the\_week[3] \# the third element of that vector, use \textasciigrave{}[]\textasciigrave{} to pull out elements in a vector}

\CommentTok{\# eg: my\_vector\_sequence[my\_vector\_sequence \textless{}= 10] \# get the values in m\_v\_s that \textless{}= 10}
\end{Highlighting}
\end{Shaded}

\#Lab 2.2 Vectors and Data Matrices

\begin{Shaded}
\begin{Highlighting}[]
\DocumentationTok{\#\#\#\#\# Matrices}
\CommentTok{\# provide vectors}
\NormalTok{Philosophers\_Stone }\OtherTok{\textless{}{-}} \FunctionTok{c}\NormalTok{(}\FloatTok{317.5}\NormalTok{, }\FloatTok{657.1}\NormalTok{)}
\NormalTok{Chamber\_of\_Secrets }\OtherTok{\textless{}{-}} \FunctionTok{c}\NormalTok{(}\FloatTok{261.9}\NormalTok{, }\FloatTok{616.9}\NormalTok{)}
\NormalTok{Prisoner\_of\_Azkaban }\OtherTok{\textless{}{-}} \FunctionTok{c}\NormalTok{(}\FloatTok{249.5}\NormalTok{, }\FloatTok{547.1}\NormalTok{)}
\NormalTok{Goblet\_of\_Fire }\OtherTok{\textless{}{-}} \FunctionTok{c}\NormalTok{(}\FloatTok{290.0}\NormalTok{, }\FloatTok{606.8}\NormalTok{)}
\NormalTok{Order\_of\_the\_Phoenix }\OtherTok{\textless{}{-}} \FunctionTok{c}\NormalTok{(}\FloatTok{292.0}\NormalTok{, }\FloatTok{647.8}\NormalTok{)}
\NormalTok{Half\_Blood\_Prince }\OtherTok{\textless{}{-}} \FunctionTok{c}\NormalTok{(}\FloatTok{301.9}\NormalTok{, }\FloatTok{632.4}\NormalTok{)}
\NormalTok{Deathly\_Hallows\_1 }\OtherTok{\textless{}{-}} \FunctionTok{c}\NormalTok{(}\FloatTok{295.9}\NormalTok{, }\FloatTok{664.3}\NormalTok{)}
\NormalTok{Deathly\_Hallows\_2 }\OtherTok{\textless{}{-}} \FunctionTok{c}\NormalTok{(}\FloatTok{381.0}\NormalTok{, }\FloatTok{960.5}\NormalTok{)}


\DocumentationTok{\#\# list values by names}
\NormalTok{box\_office }\OtherTok{\textless{}{-}} \FunctionTok{c}\NormalTok{(Philosophers\_Stone, Chamber\_of\_Secrets, Prisoner\_of\_Azkaban, Goblet\_of\_Fire, Order\_of\_the\_Phoenix, Half\_Blood\_Prince, Deathly\_Hallows\_1, Deathly\_Hallows\_2)}
\NormalTok{box\_office}
\end{Highlighting}
\end{Shaded}

\begin{verbatim}
##  [1] 317.5 657.1 261.9 616.9 249.5 547.1 290.0 606.8 292.0 647.8 301.9 632.4
## [13] 295.9 664.3 381.0 960.5
\end{verbatim}

\begin{Shaded}
\begin{Highlighting}[]
\DocumentationTok{\#\# make matric by row}
\NormalTok{harry\_potter\_matrix }\OtherTok{\textless{}{-}} \FunctionTok{matrix}\NormalTok{(box\_office, }\AttributeTok{nrow =} \DecValTok{8}\NormalTok{, }\AttributeTok{byrow =}\NormalTok{ T)}\DocumentationTok{\#\# organize it by rows}



\CommentTok{\# Provide column names}
\NormalTok{region }\OtherTok{\textless{}{-}} \FunctionTok{c}\NormalTok{(}\StringTok{"US"}\NormalTok{, }\StringTok{"non{-}US"}\NormalTok{)}
\NormalTok{region}
\end{Highlighting}
\end{Shaded}

\begin{verbatim}
## [1] "US"     "non-US"
\end{verbatim}

\begin{Shaded}
\begin{Highlighting}[]
\CommentTok{\# Provide row names}
\NormalTok{titles }\OtherTok{\textless{}{-}} \FunctionTok{c}\NormalTok{(}\StringTok{"Philosophers\_Stone"}\NormalTok{, }\StringTok{"Chamber\_of\_Secrets"}\NormalTok{, }\StringTok{"Prisoner\_of\_Azkaban"}\NormalTok{, }\StringTok{"Goblet\_of\_Fire"}\NormalTok{, }\StringTok{"Order\_of\_the\_Phoenix"}\NormalTok{, }\StringTok{"Half\_Blood\_Prince"}\NormalTok{, }\StringTok{"Deathly\_Hallows\_1"}\NormalTok{, }\StringTok{"Deathly\_Hallows\_2"}\NormalTok{)}
\NormalTok{titles}
\end{Highlighting}
\end{Shaded}

\begin{verbatim}
## [1] "Philosophers_Stone"   "Chamber_of_Secrets"   "Prisoner_of_Azkaban" 
## [4] "Goblet_of_Fire"       "Order_of_the_Phoenix" "Half_Blood_Prince"   
## [7] "Deathly_Hallows_1"    "Deathly_Hallows_2"
\end{verbatim}

\begin{Shaded}
\begin{Highlighting}[]
\CommentTok{\# Add col names to the matrices}
\FunctionTok{colnames}\NormalTok{(harry\_potter\_matrix) }\OtherTok{\textless{}{-}}\NormalTok{ region}

\CommentTok{\# Add row names to the matrices}
\FunctionTok{rownames}\NormalTok{(harry\_potter\_matrix) }\OtherTok{\textless{}{-}}\NormalTok{ titles}



\DocumentationTok{\#\#\#\#\# Row sums}
\NormalTok{global }\OtherTok{\textless{}{-}} \FunctionTok{rowSums}\NormalTok{(harry\_potter\_matrix)}
\NormalTok{global}
\end{Highlighting}
\end{Shaded}

\begin{verbatim}
##   Philosophers_Stone   Chamber_of_Secrets  Prisoner_of_Azkaban 
##                974.6                878.8                796.6 
##       Goblet_of_Fire Order_of_the_Phoenix    Half_Blood_Prince 
##                896.8                939.8                934.3 
##    Deathly_Hallows_1    Deathly_Hallows_2 
##                960.2               1341.5
\end{verbatim}

\begin{Shaded}
\begin{Highlighting}[]
\DocumentationTok{\#\#\# Add new col to the matrices by "cbind()" }
\NormalTok{all\_harry\_potter\_matrix }\OtherTok{\textless{}{-}} \FunctionTok{cbind}\NormalTok{(harry\_potter\_matrix, global) }\DocumentationTok{\#\# show new title directly}
\NormalTok{all\_harry\_potter\_matrix}
\end{Highlighting}
\end{Shaded}

\begin{verbatim}
##                         US non-US global
## Philosophers_Stone   317.5  657.1  974.6
## Chamber_of_Secrets   261.9  616.9  878.8
## Prisoner_of_Azkaban  249.5  547.1  796.6
## Goblet_of_Fire       290.0  606.8  896.8
## Order_of_the_Phoenix 292.0  647.8  939.8
## Half_Blood_Prince    301.9  632.4  934.3
## Deathly_Hallows_1    295.9  664.3  960.2
## Deathly_Hallows_2    381.0  960.5 1341.5
\end{verbatim}

\begin{Shaded}
\begin{Highlighting}[]
\CommentTok{\# rbind() to add new row to a matrices}

\CommentTok{\# colSums() calculates the col total}

\CommentTok{\# matrice[a,b] \# a is row, b is col}

\NormalTok{all\_harry\_potter\_matrix[}\DecValTok{1}\NormalTok{,}\DecValTok{3}\NormalTok{] }\DocumentationTok{\#\# 1st row, 3rd col value}
\end{Highlighting}
\end{Shaded}

\begin{verbatim}
## [1] 974.6
\end{verbatim}

\begin{Shaded}
\begin{Highlighting}[]
\NormalTok{all\_harry\_potter\_matrix[}\DecValTok{1}\SpecialCharTok{:}\DecValTok{3}\NormalTok{] }\DocumentationTok{\#\# 1st row, 3 values in that row}
\end{Highlighting}
\end{Shaded}

\begin{verbatim}
## [1] 317.5 261.9 249.5
\end{verbatim}

\begin{Shaded}
\begin{Highlighting}[]
\FunctionTok{colMeans}\NormalTok{(all\_harry\_potter\_matrix) }\CommentTok{\# all column mean}
\end{Highlighting}
\end{Shaded}

\begin{verbatim}
##       US   non-US   global 
## 298.7125 666.6125 965.3250
\end{verbatim}

\begin{Shaded}
\begin{Highlighting}[]
\FunctionTok{mean}\NormalTok{(all\_harry\_potter\_matrix[,}\DecValTok{3}\NormalTok{]) }\CommentTok{\# single column mean}
\end{Highlighting}
\end{Shaded}

\begin{verbatim}
## [1] 965.325
\end{verbatim}

\begin{Shaded}
\begin{Highlighting}[]
\CommentTok{\# rowMeans()}
\end{Highlighting}
\end{Shaded}

\hypertarget{lab-3.1-data-frames}{%
\section{Lab 3.1 Data Frames}\label{lab-3.1-data-frames}}

\begin{Shaded}
\begin{Highlighting}[]
 \FunctionTok{library}\NormalTok{(}\StringTok{"tidyverse"}\NormalTok{)}
\end{Highlighting}
\end{Shaded}

\begin{verbatim}
## -- Attaching core tidyverse packages ------------------------ tidyverse 2.0.0 --
## v dplyr     1.1.2     v readr     2.1.4
## v forcats   1.0.0     v stringr   1.5.0
## v ggplot2   3.4.2     v tibble    3.2.1
## v lubridate 1.9.2     v tidyr     1.3.0
## v purrr     1.0.1     
## -- Conflicts ------------------------------------------ tidyverse_conflicts() --
## x dplyr::filter() masks stats::filter()
## x dplyr::lag()    masks stats::lag()
## i Use the conflicted package (<http://conflicted.r-lib.org/>) to force all conflicts to become errors
\end{verbatim}

\begin{Shaded}
\begin{Highlighting}[]
\DocumentationTok{\#\# data.frame(A, B,C) \# make a data frame for three vectors}

\NormalTok{Sex }\OtherTok{\textless{}{-}} \FunctionTok{c}\NormalTok{(}\StringTok{"male"}\NormalTok{, }\StringTok{"female"}\NormalTok{, }\StringTok{"male"}\NormalTok{)}
\NormalTok{Length }\OtherTok{\textless{}{-}} \FunctionTok{c}\NormalTok{(}\FloatTok{3.2}\NormalTok{, }\FloatTok{3.7}\NormalTok{, }\FloatTok{3.4}\NormalTok{)}
\NormalTok{Weight }\OtherTok{\textless{}{-}} \FunctionTok{c}\NormalTok{(}\FloatTok{2.9}\NormalTok{, }\FloatTok{4.0}\NormalTok{, }\FloatTok{3.1}\NormalTok{)}

\NormalTok{hbirds }\OtherTok{\textless{}{-}} \FunctionTok{data.frame}\NormalTok{(Sex, Length, Weight)}
\NormalTok{hbirds}
\end{Highlighting}
\end{Shaded}

\begin{verbatim}
##      Sex Length Weight
## 1   male    3.2    2.9
## 2 female    3.7    4.0
## 3   male    3.4    3.1
\end{verbatim}

\begin{Shaded}
\begin{Highlighting}[]
\FunctionTok{names}\NormalTok{(hbirds) }\CommentTok{\# show variables names}
\end{Highlighting}
\end{Shaded}

\begin{verbatim}
## [1] "Sex"    "Length" "Weight"
\end{verbatim}

\begin{Shaded}
\begin{Highlighting}[]
\FunctionTok{dim}\NormalTok{(hbirds) }\CommentTok{\# show dimension of hbirds data}
\end{Highlighting}
\end{Shaded}

\begin{verbatim}
## [1] 3 3
\end{verbatim}

\begin{Shaded}
\begin{Highlighting}[]
\FunctionTok{str}\NormalTok{(hbirds) }\CommentTok{\# show data strucutre}
\end{Highlighting}
\end{Shaded}

\begin{verbatim}
## 'data.frame':    3 obs. of  3 variables:
##  $ Sex   : chr  "male" "female" "male"
##  $ Length: num  3.2 3.7 3.4
##  $ Weight: num  2.9 4 3.1
\end{verbatim}

\begin{Shaded}
\begin{Highlighting}[]
\CommentTok{\# Let\textquotesingle{}s use lowercase names when we create the data frame. }
\CommentTok{\# We just changed to lowercase here, but we could use any names we wish.  }

\NormalTok{hbirds }\OtherTok{\textless{}{-}} \FunctionTok{data.frame}\NormalTok{(}\AttributeTok{sex =}\NormalTok{ Sex, }\AttributeTok{length =}\NormalTok{ Length, }\AttributeTok{weight\_g =}\NormalTok{ Weight)}
\DocumentationTok{\#\#  rename{-}{-} new.name = old.name}

\CommentTok{\# rename (new = old)}
\CommentTok{\# rename(mammals, genus = "Genus", afr = "AFR", maxlife = "max. life", littersyears = "litters/year")}

\CommentTok{\# Adding a new row}
\NormalTok{new\_bird }\OtherTok{\textless{}{-}} \FunctionTok{c}\NormalTok{(}\StringTok{"female"}\NormalTok{, }\FloatTok{3.6}\NormalTok{, }\FloatTok{3.9}\NormalTok{)}
\NormalTok{new\_bird}
\end{Highlighting}
\end{Shaded}

\begin{verbatim}
## [1] "female" "3.6"    "3.9"
\end{verbatim}

\begin{Shaded}
\begin{Highlighting}[]
\NormalTok{hbirds}\OtherTok{\textless{}{-}} \FunctionTok{rbind}\NormalTok{(hbirds, new\_bird)}
\NormalTok{hbirds}
\end{Highlighting}
\end{Shaded}

\begin{verbatim}
##      sex length weight_g
## 1   male    3.2      2.9
## 2 female    3.7        4
## 3   male    3.4      3.1
## 4 female    3.6      3.9
\end{verbatim}

\begin{Shaded}
\begin{Highlighting}[]
\CommentTok{\# Adding a new col by $}
\NormalTok{hbirds}\SpecialCharTok{$}\NormalTok{neighborhood }\OtherTok{\textless{}{-}} \FunctionTok{c}\NormalTok{(}\StringTok{"lakewood"}\NormalTok{, }\StringTok{"brentwood"}\NormalTok{, }\StringTok{"lakewood"}\NormalTok{, }\StringTok{"scenic Heights"}\NormalTok{)}
\NormalTok{hbirds}
\end{Highlighting}
\end{Shaded}

\begin{verbatim}
##      sex length weight_g   neighborhood
## 1   male    3.2      2.9       lakewood
## 2 female    3.7        4      brentwood
## 3   male    3.4      3.1       lakewood
## 4 female    3.6      3.9 scenic Heights
\end{verbatim}

\begin{Shaded}
\begin{Highlighting}[]
\DocumentationTok{\#\# Writing data to file{-}{-} save hbirds with name hbirds\_data.csv}

\FunctionTok{write.csv}\NormalTok{(hbirds, }\StringTok{"hbirds\_data.csv"}\NormalTok{, }\AttributeTok{row.names =} \ConstantTok{FALSE}\NormalTok{) }\DocumentationTok{\#\#csv = common separate value}

\CommentTok{\# We use \textasciigrave{}row.names = FALSE\textasciigrave{} to avoid row numbers from printing out. }
\end{Highlighting}
\end{Shaded}

\hypertarget{lab-3.2-data-frames}{%
\section{Lab 3.2 Data Frames}\label{lab-3.2-data-frames}}

\begin{Shaded}
\begin{Highlighting}[]
\FunctionTok{getwd}\NormalTok{()}\CommentTok{\# check the working directory}
\end{Highlighting}
\end{Shaded}

\begin{verbatim}
## [1] "/Users/zhuoyawang/Desktop/GitHub/BIS15W2024_zywang/mid1 cheatsheet"
\end{verbatim}

\begin{Shaded}
\begin{Highlighting}[]
\CommentTok{\#hot\_springs \textless{}{-} read\_csv("lab3/hsprings\_data.csv") \# Load the file}
\CommentTok{\#class(hot\_springs$scientist)}
\CommentTok{\#hot\_springs$scientist \textless{}{-} as.factor(hot\_springs$scientist) \# change the class of \textquotesingle{}scientist\textquotesingle{} variable to factor}

\CommentTok{\#levels(hot\_springs$scientist) \# show level of that}


\CommentTok{\# glimpse() another way to show strucutre}
\CommentTok{\# summary() summary our data frame}
\CommentTok{\# head() first rows of data}
\CommentTok{\# tail() last rows of data}
\CommentTok{\# table(hot\_springs$scientist)\# produces counts of the number of observations in a variable}
\end{Highlighting}
\end{Shaded}

\hypertarget{lab-4.1-select-extract-variables-col}{%
\section{Lab 4.1 select() extract variables
(col)}\label{lab-4.1-select-extract-variables-col}}

\begin{Shaded}
\begin{Highlighting}[]
\CommentTok{\# select(fish, "lakeid", "scalelength") \# select(dataname, "var1", "var2") to pull out the iterest variables }

\CommentTok{\# select(fish, fish\_id:length) \#\# select from the start\_col to end\_col}

\CommentTok{\# select(fish, {-}"fish\_id", {-}"annnumber", {-}"length", {-}"radii\_length\_mm") \# minus operator is going to select everything expect the select variables.   }

\CommentTok{\# select(fish, contains("length")) extract the variables whose name contains \textquotesingle{}length\textquotesingle{}}

\CommentTok{\# select(fish, starts\_with("radii")) Select columns that start with a character string  }
\CommentTok{\# select(fish, matches("a.+er"))  a column contains a letter (in this case "a") followed by a subsequent string (in this case "er") {-}{-} \textgreater{} "annnunumber }


\CommentTok{\# select\_if(fish, is.numeric) extract numeric variables in fish dataset}

\CommentTok{\# select\_if(fish, \textasciitilde{}!is.numeric(.)) \# select if in the fish data look across all the data, do not select all numeric variables. ! is Not}

\CommentTok{\# select\_all(mammals, tolower) \#use lowercase to keep}
\end{Highlighting}
\end{Shaded}

Options to select columns based on a specific criteria include:\\
1. ends\_with() = Select columns that end with a character string\\
2. contains() = Select columns that contain a character string\\
3. matches() = Select columns that match a regular expression\\
4. one\_of() = Select columns names that are from a group of names

\hypertarget{lab-4.2-filter-extract-row-from-dataset}{%
\section{Lab 4.2 filter() -- extract row from
dataset}\label{lab-4.2-filter-extract-row-from-dataset}}

\begin{Shaded}
\begin{Highlighting}[]
\CommentTok{\# filter(fish, lakeid == "AL")\# select the rows which contain AL\#\# 2equal sign and ""}

\CommentTok{\# !=  is not equal }

\CommentTok{\# filter(fish, length \%in\% c(167, 175)) \# choose the observation with length = 167 and length = 175}

\CommentTok{\# filter(fish, between(scalelength, 2.5, 2.55)) \#filter the observation with scalelength in between (2.5 2.55)   between a range {-}{-}\textgreater{} between(variable\_name, range)}

\CommentTok{\# filter(fish, near(radii\_length\_mm, 2, tol = 0.2))\# tol = tolerance; extract observations "near" a certain value with tolerance range}

\CommentTok{\# filter(fish, lakeid == "AL" \& length \textgreater{} 350) From Al or length \textgreater{} 350 (more rows than use "and")}


\CommentTok{\# filter(fish, length \textgreater{} 400, (scalelength \textgreater{} 11 | radii\_length\_mm \textgreater{} 8)). we filter out the fish with a length over 400 and a scale length over 11 or a radii length over 8.}
\end{Highlighting}
\end{Shaded}

+\texttt{filter()} allows all of the expected operators;
i.e.~\textgreater, \textgreater=, \textless, \textless=, != (not equal),
and == (equal)

`\textbar{}' is or; `\&' is and

Rules:\\
+ \texttt{filter(condition1,\ condition2)} will return rows where both
conditions are met.\\
+ \texttt{filter(condition1,\ !condition2)} will return all rows where
condition one is true but condition 2 is not.\\
+ \texttt{filter(condition1\ \textbar{}\ condition2)} will return rows
where condition 1 or condition 2 is met.\\
+ \texttt{filter(xor(condition1,\ condition2)} will return all rows
where only one of the conditions is met, and not when both conditions
are met.

\hypertarget{lab-5.1-filter-2.0}{%
\section{Lab 5.1 --\textgreater{} filter 2.0}\label{lab-5.1-filter-2.0}}

\begin{Shaded}
\begin{Highlighting}[]
\FunctionTok{library}\NormalTok{(janitor)}
\end{Highlighting}
\end{Shaded}

\begin{verbatim}
## 
## Attaching package: 'janitor'
\end{verbatim}

\begin{verbatim}
## The following objects are masked from 'package:stats':
## 
##     chisq.test, fisher.test
\end{verbatim}

\begin{Shaded}
\begin{Highlighting}[]
\CommentTok{\# clean\_names() change variables\textquotesingle{} names to lower cases}

\CommentTok{\# primate \textless{}{-} filter(new\_mammals, genus \%in\% c("Lophocebus" , "Erythrocebus"  ,"Macaca")) \#\# within is \%in\% 选择有L,E,M的rows}
\end{Highlighting}
\end{Shaded}

\hypertarget{lab-5.2-pipes-arrange-mutate-and-if_else}{%
\section{Lab 5.2 --\textgreater{} Pipes, arrange(), mutate(), and
if\_else()}\label{lab-5.2-pipes-arrange-mutate-and-if_else}}

\begin{enumerate}
\def\labelenumi{\arabic{enumi}.}
\tightlist
\item
  Use pipes to connect functions in dplyr.\\
\item
  Use \texttt{arrange()} to order dplyr outputs.\\
\item
  Use \texttt{mutate()} to add columns in a dataframe.\\
\item
  Use \texttt{mutate()} and \texttt{if\_else()} to replace values in a
  dataframe.
\end{enumerate}

pipes

\begin{Shaded}
\begin{Highlighting}[]
\CommentTok{\# select(fish, lakeid, scalelength)}
\CommentTok{\# filter(fish, lakeid == "AL")}

\DocumentationTok{\#\#\#\#\#\# Pipe can call the data at one time}

\CommentTok{\# fish\%\textgreater{}\%}
\CommentTok{\# select(lakeid, scalelength)\%\textgreater{}\%}
\CommentTok{\# filter(lakeid == "AL")}


\DocumentationTok{\#\#\# example}

\CommentTok{\#fish \%\textgreater{}\% \#work with the fish data}
\CommentTok{\#  select( lakeid, radii\_length\_mm) \%\textgreater{}\% \# pull out variables of interest}
\CommentTok{\#  filter(lakeid == "AL"|lakeid ==  "AR")\%\textgreater{}\% \#only these lakes}
\CommentTok{\#  filter(between(radii\_length\_mm, 2,4)) \#sort to make easier to read}
\end{Highlighting}
\end{Shaded}

arrange() function

\begin{Shaded}
\begin{Highlighting}[]
\DocumentationTok{\#\#\#\#\#\# arrange() like a sord command, and it always show in a ascending order(small to large)}

\CommentTok{\#fish \%\textgreater{}\% }
\CommentTok{\#  select(lakeid, scalelength) \%\textgreater{}\% }
\CommentTok{\#  arrange(scalelength)}

\DocumentationTok{\#\#\#\#\#\# arrange(dec()) make it to be descending order (large to small)}
\end{Highlighting}
\end{Shaded}

mutate() function helps creating a new col from the exsiting variables

\begin{Shaded}
\begin{Highlighting}[]
\CommentTok{\#fish \%\textgreater{}\%  in the fish dataset }
\CommentTok{\#  mutate(length\_mm = length*10) \%\textgreater{}\% create a new col with name length\_mm that leng*10}
\CommentTok{\#  select(fish\_id, length, length\_mm) select three cols from the dataset}



\DocumentationTok{\#\#\# mutate\_all() is helpful in cleaning data }
\CommentTok{\#mammals \%\textgreater{}\%}
\CommentTok{\#  mutate\_all(tolower) make the obs to be all lowercase (not the variables names)}


\DocumentationTok{\#\#\# use across() to specify the cols we want to clean}
\CommentTok{\#mammals \%\textgreater{}\% }
\CommentTok{\#  mutate(across(c("order", "family"), tolower))}



\DocumentationTok{\#\#\# ifelse() {-}{-}\textgreater{} replace {-}999.00 with NA, and others remain the same value as newborn}
\CommentTok{\#mammals \%\textgreater{}\% }
\CommentTok{\#  select(genus, species, newborn) \%\textgreater{}\%}
\CommentTok{\#  mutate(newborn\_new = ifelse(newborn == {-}999.00, NA, newborn))\%\textgreater{}\% }
\CommentTok{\#  arrange(newborn)}
\end{Highlighting}
\end{Shaded}

Lab 6.1 same as 5.2

\hypertarget{lab-6.2-tabyl-a-version-of-table-and-produces-counts-but-also-percentages}{%
\section{Lab 6.2 tabyl() a version of table and produces counts but also
percentages}\label{lab-6.2-tabyl-a-version-of-table-and-produces-counts-but-also-percentages}}

\begin{Shaded}
\begin{Highlighting}[]
\FunctionTok{library}\NormalTok{(}\StringTok{"janitor"}\NormalTok{)}
\CommentTok{\# tabyl(dataset, variable\_name)}
\end{Highlighting}
\end{Shaded}

\hypertarget{lab-7.1-summarize-group_by-and-n_distinct}{%
\section{Lab 7.1 summarize(), group\_by() and
n\_distinct()}\label{lab-7.1-summarize-group_by-and-n_distinct}}

\texttt{summarize()} will produce summary statistics for a given
variable in a data frame.

\begin{Shaded}
\begin{Highlighting}[]
\CommentTok{\#install.packages("skimr")}
\FunctionTok{library}\NormalTok{(}\StringTok{"skimr"}\NormalTok{)}

\CommentTok{\# msleep\%\textgreater{}\%}
\CommentTok{\#  filter(bodywt \textgreater{} 200)\%\textgreater{}\% \# extract the rows with bodywt \textgreater{} 200}
\CommentTok{\#  summarize(mean\_sleep\_lg = mean(sleep\_total)) \# calculate the mean of total sleep of animals whose body weights are over 200}


\DocumentationTok{\#\#\#\#\# example}
\CommentTok{\# msleep\%\textgreater{}\%}
\CommentTok{\#  filter(bodywt \textgreater{} 200)\%\textgreater{}\%}
\CommentTok{\#  summarize(mean\_sleep\_lg = mean(sleep\_total),}
\CommentTok{\#            min\_sleep\_lg = min(sleep\_total),}
\CommentTok{\#            max\_sleep\_lg = max(sleep\_total),}
\CommentTok{\#            sd\_sleep\_lg = sd(sleep\_total),}
\CommentTok{\#            total = n()) \# total number of observations}
\end{Highlighting}
\end{Shaded}

\texttt{n\_distinct()} is a very handy way of cleanly presenting the
number of distinct observations.

\begin{Shaded}
\begin{Highlighting}[]
\CommentTok{\#msleep\%\textgreater{}\%}
\CommentTok{\#  summarize(n\_genera = n\_distinct(genus))\# this is going to count the number of genera in msleep}


\CommentTok{\# n\_distinct() makes a integer and summarize makes it to be data frame.}

\CommentTok{\# n\_distinct() is similar as unique() }
\end{Highlighting}
\end{Shaded}

`group\_by()' providing what we want by the diffeent of `vore'

\begin{Shaded}
\begin{Highlighting}[]
\CommentTok{\#msleep \%\textgreater{}\%}
\CommentTok{\#  group\_by(vore) \%\textgreater{}\% \#we are grouping by feeding ecology, a categorical variable}
\CommentTok{\#  summarize(min\_bodywt = min(bodywt),}
\CommentTok{\#           max\_bodywt = max(bodywt),}
\CommentTok{\#            mean\_bodywt = mean(bodywt),}
\CommentTok{\#           total=n())}
\end{Highlighting}
\end{Shaded}

\hypertarget{lab-7.2-summarize-practice-count-across}{%
\section{Lab 7.2 summarize practice, count(),
across()}\label{lab-7.2-summarize-practice-count-across}}

\begin{Shaded}
\begin{Highlighting}[]
\DocumentationTok{\#\# pull out all obs with a number = remvoe NAs}

\CommentTok{\#penguins\%\textgreater{}\%}
\CommentTok{\#  filter(!is.na(body\_mass\_g))\%\textgreater{}\%\#pull out all of the observations with a number}
\CommentTok{\#  group\_by(island)\%\textgreater{}\%}
\CommentTok{\#  summarize(n=n(), mean\_body\_mass = mean(body\_mass\_g))}

\DocumentationTok{\#\#\#\# remove warning maessage by .group = \textquotesingle{}keep\textquotesingle{}}

\CommentTok{\#penguins \%\textgreater{}\% }
\CommentTok{\#  group\_by(species, island) \%\textgreater{}\% }
\CommentTok{\#  summarize(n=n(),.groups= \textquotesingle{}keep\textquotesingle{})\#the .groups argument here just prevents a warning message}
\end{Highlighting}
\end{Shaded}

\texttt{count()} is an easy way of determining how many observations you
have within a column. It acts like a combination of \texttt{group\_by()}
and \texttt{n()}.

\begin{Shaded}
\begin{Highlighting}[]
\CommentTok{\#penguins\%\textgreater{}\%}
\CommentTok{\#  group\_by(island)\%\textgreater{}\%}
\CommentTok{\#  summarize(n=n()) summary the the numeber of obs by groups}

\DocumentationTok{\#\#\# same result computed by count()}

\CommentTok{\#penguins \%\textgreater{}\% }
\CommentTok{\#  count(island, sort = T) \#sort=T sorts the column in descending order}




\DocumentationTok{\#\#\#\#\#\# for multi variables with counts}
\CommentTok{\#penguins \%\textgreater{}\% }
\CommentTok{\#  count(island, species, sort = T) \# sort=T will arrange in descending order}

\DocumentationTok{\#\#\#\#\# same output with the following}

\CommentTok{\#penguins \%\textgreater{}\%}
\CommentTok{\#  tabyl(island, species)}


\DocumentationTok{\#\#\#\#\# compute the distinct obs for three variables}

\CommentTok{\#penguins \%\textgreater{}\%}
\CommentTok{\#  summarize(across(c(species, island, sex), n\_distinct))\# n\_dis counts the number of unique}



\DocumentationTok{\#\#\#\# For continues variables}
\CommentTok{\#penguins \%\textgreater{}\%}
\CommentTok{\#  summarize(across(contains("mm"), mean, na.rm=T)) \# compute mean values of all varibales with \textquotesingle{}mm\textquotesingle{} without NA}

\DocumentationTok{\#\#\#\# remove warning }

\CommentTok{\#penguins \%\textgreater{}\%}
\CommentTok{\#  summarize(across(contains("mm"), \textbackslash{}(x) mean(x, na.rm = TRUE)))\#use this to correct the error}
\end{Highlighting}
\end{Shaded}


\end{document}
